\usetheme[progressbar=frametitle,block=fill]{metropolis}
\setbeamertemplate{theorems}[numbered]
\RequirePackage{appendixnumberbeamer}

\RequirePackage{xcolor}
\RequirePackage{algorithm}
\RequirePackage{algpseudocode}
\RequirePackage{amsfonts}
\RequirePackage{amsmath}
\RequirePackage{amssymb}
\RequirePackage{amsthm}
\RequirePackage{bm}
\RequirePackage{bigdelim}
\RequirePackage{bigstrut}
\RequirePackage{bookmark}
\RequirePackage{booktabs}
\RequirePackage[scale=2]{ccicons}
\RequirePackage{cprotect}
\RequirePackage[UTF8]{ctex}
\RequirePackage{etoolbox}
\RequirePackage{extpfeil}
\RequirePackage{fancyhdr}
\RequirePackage{hyperref}
\RequirePackage{ifthen}
\RequirePackage{subcaption}
\RequirePackage{tabularx}
\RequirePackage{tikz}
\RequirePackage{url}
\RequirePackage{xspace}
\RequirePackage{longtable}
\RequirePackage{mathtools}
\RequirePackage{multirow}
\RequirePackage{multicol}
\RequirePackage{geometry}
\RequirePackage{xfp}
\RequirePackage{pgfplots}
\RequirePackage{ulem}

\usepgfplotslibrary{dateplot}
\usepgfplotslibrary{groupplots}

\usetikzlibrary{automata}
\usetikzlibrary{backgrounds}
\usetikzlibrary{calc}
\usetikzlibrary{intersections}
\usetikzlibrary{math}
\usetikzlibrary{positioning}
\usetikzlibrary{shapes.geometric}
\usetikzlibrary{shapes.misc}


% Font setting
\RequirePackage{eulervm}
\RequirePackage{fontspec}
\setsansfont{Fira Sans}
\setmonofont{Fira Code}[Contextuals=Alternate]


% Image
\newcommand{\includeimage}[3][scale=1]{
	\begin{figure}
		\centering
		\includegraphics[#1]{image/#2}
		\caption{#3}
		\label{fig:#2}
	\end{figure}
}

\newcommand{\includetikzimage}[2]{
	\begin{figure}
		\centering
		\input{image/#1}
		\caption{#2}
		\label{fig:#1}
	\end{figure}
}


% Code style
\RequirePackage{listings}
\RequirePackage{lstfiracode}
\lstdefinestyle{common}{
	style=FiraCodeStyle,
	belowcaptionskip=1\baselineskip,
	breaklines=true,
	xleftmargin=\parindent,
	showstringspaces=true,
	numbers=left,
	numberstyle=\ttfamily\small,
	basicstyle=\ttfamily,
	stepnumber=1,
	frame=single
}

\lstdefinestyle{code}{
	style=common,
	keywordstyle=\bfseries\color{green!40!black},
	commentstyle=\itshape\color{red!80!black},
	identifierstyle=\color{blue},
	stringstyle=\color{purple!40!black}
}

\lstdefinestyle{c++}{
	style=code,
	language=C++
}

\lstdefinestyle{c}{
	style=code,
	language=C
}

\lstdefinestyle{py}{
	style=code,
	language=Python
}

\lstdefinestyle{java}{
	style=code,
	language=Java
}

\newcommand{\includecode}[2][c++]{
	\vspace{0.3cm}
	\lstset{style=#1}
	\lstinputlisting[label={code:#2}]{code/#2}
	\vspace{0.3cm}
}

\lstdefinestyle{pascal}{
	style=code,
	language=Pascal
}


% Bib
\newcommand{\listofbib}[1][references]{
	\nocite{*}
	\bibliographystyle{plain}
	\bibliography{bib/#1}
}


% Throrem
\renewcommand\qedsymbol{\(\blacksquare\)}


% User defined math command
\newcommand{\lcm}{\operatorname{lcm}}
\newcommand{\nequiv}{~{\equiv}\llap{/\,}~}
\newcommand{\subjectto}{~s.t.~}


% Others
\newcommand{\forcenewline}{\leavevmode\newline}
\newcommand{\questions}{
	\begin{frame}[standout]
		Questions?
	\end{frame}
}


\newcommand{\drawcsys}[5][0]{% [step]{xmin}{xmax}{ymin}{ymax}
	\def\vstep{#1}
	\def\vxmin{#2}
	\def\vxmax{#3}
	\def\vymin{#4}
	\def\vymax{#5}

	\ifstrequal{#1}{0}{}{%
		\draw[help lines,step=\vstep] (\vxmin,\vymin) grid (\vxmax,\vymax);
	}
	\draw[->,thick] (\vxmin,0) -- (\vxmax,0) node[anchor=north] {\(x\)};
	\draw[->,thick] (0,\vymin) -- (0,\vymax) node[anchor=east]  {\(y\)};
}
